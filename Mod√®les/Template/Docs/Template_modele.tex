\documentclass[a4paper,12pt]{article}
\usepackage[utf8]{inputenc}
\usepackage[T1]{fontenc}
\usepackage[francais]{babel}
\usepackage{amsmath}

\begin{document}
\begin{center}
\fbox{\LARGE \textbf{Titre Modèle}}
\end{center}
\bigskip

\large{\underline{\textbf{Description du modèle}:}}\\\\\normalsize
Le modèle décrit ci-après est tiré de \cite{Doc1}. \\

\underline{\textbf Nomenclature:}
\begin{itemize}
	\item $ X_{1} $ la grandeur 1 en $ Unité 1 $
	\item $ X_{2} $ la grandeur 2 en $ Unité 2 $\\
\end{itemize}

\underline{\textbf Equations:}
\begin{itemize}
	\item $ X_{1}^{2} = X_{2} $\\\\
\end{itemize}

\large{\underline{\textbf{Cahier des Charges}:}}\\\\\normalsize
Dans cet exemple, on fixe $ X_{1} $ ainsi qu'un certain nombre de grandeurs désignées comme paramètres et on cherche les variables de décision qui minimisent $ l'objectif $ en respectant les contraintes.\\

\begin{center}
\begin{tabular}{|c|c|c|c|c|}
 \hline
 \multicolumn{5}{|c|}{Variables de Décision} \\
 \hline
 Paramètre & Valeur min & Valeur max & Valeur initiale & Unité\\
 \hline
 $ X_{1} $ & $ 0.1 $ & $ 1.0 $ & $ 0.5 $ & $ Unité 1 $\\
 \hline
 $ X_{2} $ & \multicolumn{3}{c|}{$ 1.0 $} & $ Unité 2 $ \\
 \hline
\end{tabular}
\end{center}

\begin{center}
\begin{tabular}{|c|c|c|c|}
 \hline
 \multicolumn{4}{|c|}{Sorties} \\
 \hline
 Paramètre & Type & Valeur & Unité\\
 \hline
 $ X_{1} $ & $ Fixe $ & $ 10 $ & $ Unité 1 $\\
 \hline
 $ X_{2} $ & $ Libre $ & $ - $ & $ Unité 2 $\\
 \hline
\end{tabular}
\end{center}

\underline{\textbf Fonction Objectif:}\\
\[ f_{obj}(V) = X_{1}^{2} + \sqrt{X_{2}} \]\\

\large{\underline{\textbf{Test de Fiabilité}:}}\\

\normalsize
Afin de vérifier la validité du modèle proposé, il convient de tester ce dernier avec plusieurs sets de valeurs. Vous trouverez ci-après un ensemble de valeurs d'entrée et les résultats attendus sur la base des valeurs de \cite{Doc1}.\\

\begin{center}
\begin{tabular}{|c|c|c|c|c|}
 \hline
 Numéro du set & Set 1 & Set 2 & Set 3 & Set 4\\
 \hline
 $ X_{1} $ & $ X_{1}^{1} $ & $ X_{1}^{2} $ & $ X_{1}^{3} $ & $ X_{1}^{4} $ \\
 \hline
 $ X_{2} $ & $ X_{2}^{1} $ & $ X_{2}^{2} $ & $ X_{2}^{3} $ & $ X_{2}^{4} $ \\
 \hline
\end{tabular}
\end{center}


\bibliography{Biblio}
\bibliographystyle{ieeetr}

\end{document}